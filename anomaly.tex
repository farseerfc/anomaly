%\documentclass[CJK]{beamer}  
%\usetheme[left,width=5em]{Goettingen}
\usepackage{xeCJK}
\makeatletter
\newcommand{\newinfo}[1]{}
\makeatother

\usepackage{fontspec,xunicode,xltxtra,listings}

\usepackage[caption=false,font=footnotesize]{subfig}
\usepackage{tikz,xcolor}
\usetikzlibrary{shapes,arrows,shadows,mindmap,backgrounds}


\setbeamercovered{transparent}
\setbeamertemplate{items}[circle]
\renewcommand{\figurename}{図}
\newtheorem{definationfc}{定義}

\usepackage[overlap,CJK]{ruby}
\usepackage{multicol}
%\usepackage[sort&compress]{natbib}
%\usepackage{chapterbib}

\newcommand{\backupbegin}{
   \newcounter{framenumberappendix}
   \setcounter{framenumberappendix}{\value{framenumber}}
}
\newcommand{\backupend}{
   \addtocounter{framenumberappendix}{-\value{framenumber}}
   \addtocounter{framenumber}{\value{framenumberappendix}} 
}

\newcommand{\fcshadow}[1]{%
\tikz\path[anchor=north west]
node[fill opacity=0.1] at (0.05em,-0.05em) {#1}
node[fill opacity=0.1] at (0.03em,-0.05em) {#1}
node[fill opacity=0.1] at (0.05em,-0.03em) {#1}
node[fill opacity=0.1] at (0.05em,-0.07em) {#1}
node[fill opacity=0.1] at (0.07em,-0.05em) {#1}
node at (0pt,0pt) {#1};}


\setCJKmainfont{MS PGothic}
\setCJKsansfont{Meiryo UI} %[Mapping=tex-text]
\setCJKmonofont{Courier New}
%\usefonttheme{serif}
\setmainfont{Droid Serif}
\setsansfont{Droid Sans} %[Mapping=tex-text]
\setmonofont{Droid Sans Mono}

\def\beamer@linkspace#1{%
  \begin{pgfpicture}{0pt}{-1.5pt}{#1}{5.5pt}
    \pgfsetfillopacity{0}
    \pgftext[x=0pt,y=-1.5pt]{.}
    \pgftext[x=#1,y=5.5pt]{.}
  \end{pgfpicture}}

\lstset{tabsize=4, %
  frame=shadowbox, %把代码用带有阴影的框圈起来
  commentstyle=\color{red!50!green!50!blue!50},%浅灰色的注释
  rulesepcolor=\color{red!20!green!20!blue!20},%代码块边框为淡青色
  keywordstyle=\color{blue!90}\bfseries, %代码关键字的颜色为蓝色,粗体
  showstringspaces=false,%不显示代码字符串中间的空格标记
  stringstyle=\ttfamily, % 代码字符串的特殊格式
  keepspaces=true, %
  breakindent=22pt, %
  numbers=left,%左侧显示行号
  stepnumber=1,%
  numberstyle=\tiny, %行号字体用小号
  basicstyle=\footnotesize, %
  showspaces=false, %
  flexiblecolumns=true, %
  breaklines=true, %对过长的代码自动换行
  breakautoindent=true,%
  breakindent=4em, %
  aboveskip=1em, %代码块边框
  %% added by http://bbs.ctex.org/viewthread.php?tid=53451
  fontadjust,
  captionpos=t,
  framextopmargin=2pt,framexbottommargin=2pt,abovecaptionskip=-3pt,belowcaptionskip=3pt,
  xleftmargin=4em,xrightmargin=4em, % 设定listing左右的空白
  texcl=true,
  % 设定中文冲突,断行,列模式,数学环境输入,listing数字的样式
  extendedchars=false,columns=flexible,mathescape=true
  numbersep=-1em
}

%%%%%%%%%%%%%%%%%%%%%%%%%%%%%%%%%%%%%%%%%%%%%%%%%%%%%%%%%%%%%%%%%%%%%%%%

\title[Active Refinement of Clone Anomaly Reports]{Active Refinement of Clone Anomaly Reports}

\subtitle[MD 輪講]{MD 輪講}

\author[大阪大学大学院CS専攻\quad{}楊 嘉晨]{修士課程1年\quad{}楊 嘉晨}
\institute[楠本研]{大阪大学大学院 コンピュータサイエンス専攻 楠本研究室}
\date{2012年7月4日(火)}

\pgfdeclareimage[height=0.618cm]{logo}{figure/logo.pdf}
\logo{\pgfuseimage{logo}}

\mode<article>{\providecommand{\imageheight}{0.2\textheight}}
\mode<presentation>{\providecommand{\imageheight}{0.4\textheight}}

%%%%%%%%%%%%%%%%%%%%%%%%%%%%%%%%%%%%%%%%%%%%%%%%%%%%%%%%%%%%%%%%%%%%%%%%
\begin{document} 

\mode<presentation>{

\providecommand{\newblock}{\\}
\providecommand{\toprule}{\hline}
\providecommand{\midrule}{\hline}
\providecommand{\bottomrule}{\hline}
\providecommand{\itemtitle}[1]{\item \alert{#1} \quad{} }
}

\mode<article>{
\renewenvironment{columns}{\begin{multicols}{2}}{\end{multicols}\\}
\renewenvironment{column}[1]{}{}
}

\XeTeXlinebreaklocale "jp"  
\XeTeXlinebreakskip = 0pt plus 1pt 

\frame{\titlepage} 
\mode<article>{\maketitle}

\begin{frame}<trans|beamer>[shrink=10]{目次}
\tableofcontents[sectionstyle=show/hide,subsectionstyle=hide,subsubsectionstyle=hide] 
\end{frame}


\AtBeginSubsection[]{
  \begin{frame}<trans|beamer>{\insertsection}

\begin{center}
\Huge\insertsubsection
%\tableofcontents[sectionstyle=show/hide,subsectionstyle=show/shaded/hide]
\end{center}

  \end{frame}
}


\begin{frame}{123456789012345678901234567890}
12345678901234567890123456789012345678901234567890
2345678901234567890123456789012345678901234567890
34\fcshadow{567890123}4567890123456789012345678901234567890
45678901234567890123456789012345678901234567890
5678901234567890123456789012345678901234567890
678901234567890123456789012345678901234567890
78901234567890123456789012345678901234567890
8901234567890123456789012345678901234567890
901234567890123456789012345678901234567890
01234567890123456789012345678901234567890
1234567890123456789012345678901234567890
234567890123456789012345678901234567890
34567890123456789012345678901234567890
4567890123456789012345678901234567890
\end{frame}

\begin{frame}
\begin{block}{aaa}
aaa
\end{block}

\begin{itemize}
\item 12
\item 34
\end{itemize}
\begin{enumerate}
\item 12
\item 34
\end{enumerate}
\end{frame}

\begin{frame}{\fcshadow{Example}}
%\newcommand\fcshadow[2]{
%\begin{tikzpicture}\node[black!50] at (1pt,1pt) {#1};\node at (0pt,0pt) {#2};\end{tikzpicture}
%}

\fcshadow{Example}
\fcshadow{\Large Example}
\end{frame}
%%%%%%%%%%%%%%%%%%%%%%%%%%%%%%%%%%%%%%%%%%%%%%%%%%%%%%%%%%%%%%%%%%%%%%%%
\end{document}
